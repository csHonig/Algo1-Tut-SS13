\documentclass{beamer}

\usepackage{sdq/templates/beamerthemekit}

\usepackage[utf8]{inputenc}
\usepackage[ngerman]{babel}
\usepackage[TS1,T1]{fontenc}
\usepackage{array}
\usepackage{xspace}
\usepackage{xcolor}
\usepackage{listings}
\usepackage[scaled]{beramono}

\beamertemplatenavigationsymbolsempty

% vertical table padding
\renewcommand{\arraystretch}{1.5}

\definecolor{colkeywords}{rgb}{0,0,0.4}
\definecolor{colcomment}{rgb}{0,0.5,0}
\definecolor{colstring}{rgb}{0.5,0,0}
\definecolor{colconst}{rgb}{0.9,0.1,0.8}
\definecolor{collinenums}{rgb}{0.5,0.5,0.5}
\definecolor{coldigit}{rgb}{0.9,0.1,0.8}

\lstset{
  basicstyle=\normalsize\ttfamily,
  numbers=left,
  numberstyle=\scriptsize,
  numbersep=-5pt,
  tabsize=4,
  framexleftmargin=1mm,
  xleftmargin=10mm,
  escapeinside={\%*}{*)},
  commentstyle=\color{colcomment},
  keywordstyle=\color{colkeywords}\bfseries,
  stringstyle=\color{colstring},
  numberstyle=\color{collinenums},
  mathescape,
  literate=%
    {0}{{{\color{coldigit}0}}}1
    {1}{{{\color{coldigit}1}}}1
    {2}{{{\color{coldigit}2}}}1
    {3}{{{\color{coldigit}3}}}1
    {4}{{{\color{coldigit}4}}}1
    {5}{{{\color{coldigit}5}}}1
    {6}{{{\color{coldigit}6}}}1
    {7}{{{\color{coldigit}7}}}1
    {8}{{{\color{coldigit}8}}}1
    {9}{{{\color{coldigit}9}}}1
    {Ö}{{\"O}}1
    {Ä}{{\"A}}1
    {Ü}{{\"U}}1
    {ß}{{\ss}}2
    {ü}{{\"u}}1
    {ä}{{\"a}}1
    {ö}{{\"o}}1
}

\lstdefinelanguage{Pseudo}
{keywords={Function,Procedure,return,assert,while,for,to,downto,if,then,else,Pointer,Class,invariant,this,NIL,of,Array},%
emph={FALSE,TRUE},
emphstyle=\color{colconst},
sensitive=true,%
comment=[l]{//},%
string=[b]",%
}


\newcommand{\N}{\mathbb{N}}

%\usetheme[deutsch]{KIT}
\author{Simon Bischof (simon.bischof2@student.kit.edu)}
\title{Tutorium Algorithmen 1}
\institute{Institut für theoretische Informatik, Prof. Sanders}
\titleimage{title.png}

\begin{document}

\shorthandoff{"}
\lstset{language=Pseudo}

\begin{frame}
  \titlepage
\end{frame}

\begin{frame}
\frametitle{Wahlen der Studierendenschaft}
Geht wählen!
\begin{itemize}
\item Wahl des Studierendenparlaments und der Fachschaftsvorstände
\item Keine Bindung an bestimmte Wahlurnen
\item Noch bis Freitag, 14.6., 15:00 Uhr an den meisten Urnen; bis 16:00 Uhr in der Mensa
\end{itemize}
\end{frame}

\begin{frame}
\frametitle{Probeklausur}
Gut:
\begin{itemize}
\item Mastertheorem
\item Hashtabellen
\item DAG
\end{itemize}
Weniger gut:
\begin{itemize}
\item unbeschränkte Felder
\item rectMult
\item Datenstrukturinvariante doppelt-verkettete Liste / Heap-Eigenschaft
\end{itemize}
\end{frame}

\begin{frame}
\frametitle{Sortierte Folgen}
\begin{itemize}
\item Folge $\langle e_1,e_2,\ldots,e_n\rangle$ mit $e_1\leq e_2\leq \ldots\leq e_n$\pause
\item wichtigste Funktion: M.locate(k) = address of min $\{e\in M\mid e\geq k\}$\pause
\item hier: sortierte, zyklische Liste mit Dummy $\infty$ + Navigationsdatenstruktur\pause
\item statische Variante: sortiertes Array
\end{itemize}
\end{frame}

\begin{frame}
\frametitle{Operationen}
\begin{itemize}
\item O(log n): insert, remove, update, locate
\item O(1): min, max
\item O(log n+|result|): rangeSearch
\item O(n): (re)build\pause
\item Weitere Möglichkeiten: concat, split, rank, select, rangeSize, \ldots
\end{itemize}
\end{frame}

\begin{frame}
\frametitle{Binärer Suchbaum}
\begin{itemize}
\item Blätter: Elemente der sortierten Folge
\item innere Knoten: (k:Element, l:Teilbaum, r:Teilbaum)\pause
\item über l erreichbare Blätter haben Werte $\leq$ k
\item über r erreichbare Blätter haben Werte > k\pause
\item locate einfach zu implementieren
\end{itemize}
\end{frame}

\begin{frame}
\frametitle{Kreativaufgabe}
Gegeben sei ein binärer Suchbaum, der auch in den inneren Knoten Elemente speichert. Jeder
Knoten habe drei Zeiger, zwei Kindzeiger und einen Zeiger auf den Elternknoten (u.U. sind einige
davon Nullzeiger). Die Elemente sind jedoch nicht zusätzlich in einer Liste enthalten, d.h. das
nächstgrößere Element kann nicht direkt gefunden werden. Es sei ein $\infty$-Dummy enthalten.
\begin{itemize}
\item Implementiere find(x), also eine Funktion die einen Schlüssel x
nimmt und entweder das passende Element aus der Datenstruktur zurückgibt, oder $\bot$ falls
kein passendes Element enthalten ist. Die Laufzeit sollte in O(Baumtiefe) liegen.
\item Implementiere locate(x) wie aus der Vorlesung. Die Laufzeit sollte
in O(Baumtiefe) liegen.
\item Implementiere locate(x) wie aus der Vorlesung. Die Baumknoten
haben in diesem Fall aber keine Zeiger mehr auf ihre Elternknoten, und es soll nur O(1)
zusätzlicher Speicher verwendet werden. Die Laufzeit sollte in O(Baumtiefe) liegen.
\end{itemize}
\end{frame}

\begin{frame}
\frametitle{Laufzeit von locate}
\begin{itemize}
\item O(Höhe)\pause
\item Balancierter Suchbaum: worst case O(log n)
\item Für entartete Bäume aber O(n)\pause
\item Sortieren mit binären Suchbäumen?
\end{itemize}
\end{frame}

\begin{frame}
\frametitle{(a,b)-Bäume}
\begin{itemize}
\item a$\geq$2, b$\geq$2a-1
\item Blätter wie vorher; allerdings gleiche Tiefe
\item innere Knoten mit Ausgangsgrad a..b
\item Wurzel hat Ausgangsgrad 2..b (1 für $\langle\rangle$)\pause
\item bei Ausgangsgrad d gibt es d-1 "Splitter"
\end{itemize}
\end{frame}

\begin{frame}
\frametitle{Locate}
\begin{itemize}
\item Code siehe VL\pause
\item height=$h\leq 1+\lfloor\log_a\frac{n+1}{2}\rfloor$
\item Laufzeit O(b$\cdot$height)
\item Für $\{a,b\}\subseteq O(1)$ ist dies in O(log n)
\end{itemize}
\end{frame}

\begin{frame}
\frametitle{Insert}
\begin{itemize}
\item Code siehe VL
\item an der richtigen Stelle einfügen\pause
\item falls Knoten voll: spalten, Trennelement nach oben durchreichen
\item evtl. rekursiv weiter
\end{itemize}
\end{frame}

\begin{frame}
\frametitle{Delete}
\begin{itemize}
\item Code siehe VL
\item Element aus Blattliste löschen
\item Splitter entfernen\pause
\item bei Unterlauf mit Nachbarknoten fusionieren (falls möglich)\pause
\item sonst balancieren
\end{itemize}
\end{frame}
\end{document}