\documentclass{beamer}

\usepackage{sdq/templates/beamerthemekit}

\usepackage[utf8]{inputenc}
\usepackage[ngerman]{babel}
\usepackage[TS1,T1]{fontenc}
\usepackage{array}
\usepackage{xspace}
\usepackage{xcolor}
\usepackage{listings}
\usepackage[scaled]{beramono}

\beamertemplatenavigationsymbolsempty

% vertical table padding
\renewcommand{\arraystretch}{1.5}

\definecolor{colkeywords}{rgb}{0,0,0.4}
\definecolor{colcomment}{rgb}{0,0.5,0}
\definecolor{colstring}{rgb}{0.5,0,0}
\definecolor{colconst}{rgb}{0.9,0.1,0.8}
\definecolor{collinenums}{rgb}{0.5,0.5,0.5}
\definecolor{coldigit}{rgb}{0.9,0.1,0.8}

\lstset{
  basicstyle=\normalsize\ttfamily,
  numbers=left,
  numberstyle=\scriptsize,
  numbersep=-5pt,
  tabsize=4,
  framexleftmargin=1mm,
  xleftmargin=10mm,
  escapeinside={\%*}{*)},
  commentstyle=\color{colcomment},
  keywordstyle=\color{colkeywords}\bfseries,
  stringstyle=\color{colstring},
  numberstyle=\color{collinenums},
  mathescape,
  literate=%
    {0}{{{\color{coldigit}0}}}1
    {1}{{{\color{coldigit}1}}}1
    {2}{{{\color{coldigit}2}}}1
    {3}{{{\color{coldigit}3}}}1
    {4}{{{\color{coldigit}4}}}1
    {5}{{{\color{coldigit}5}}}1
    {6}{{{\color{coldigit}6}}}1
    {7}{{{\color{coldigit}7}}}1
    {8}{{{\color{coldigit}8}}}1
    {9}{{{\color{coldigit}9}}}1
    {Ö}{{\"O}}1
    {Ä}{{\"A}}1
    {Ü}{{\"U}}1
    {ß}{{\ss}}2
    {ü}{{\"u}}1
    {ä}{{\"a}}1
    {ö}{{\"o}}1
}

\lstdefinelanguage{Pseudo}
{keywords={Function,return,assert,while,for,to,downto,if,then,else},%
emph={FALSE,TRUE},
emphstyle=\color{colconst},
sensitive=true,%
comment=[l]{//},%
string=[b]",%
}


\newcommand{\N}{\mathbb{N}}

%\usetheme[deutsch]{KIT}
\author{Simon Bischof (simon.bischof2@student.kit.edu)}
\title{Tutorium Algorithmen 1}
\institute{Institut für theoretische Informatik, Prof. Sanders}
\titleimage{title.png}

\begin{document}

\shorthandoff{"}
\lstset{language=Pseudo}

\begin{frame}
  \titlepage
\end{frame}

\begin{frame}
\frametitle{Vorstellung}
Zu meiner Person:\pause
\begin{itemize}
\item Simon Bischof
\item Bachelor Informatik
\item 6. Semester
\end{itemize}
\end{frame}

\begin{frame}
\frametitle{Kontakt}
\begin{itemize}
\item E-Mail: simon.bischof2@student.kit.edu \pause
\item bei inhaltlichen und sonstigen Fragen zu Algorithmen 1 \pause
\item Liste zum Eintragen der Mailadresse
\item für kurzfristige Informationen \pause
\item Folien sind auf \textcolor{blue}{\underline{\url{https://github.com/ratefuchs/Algo1-Tut-SS13}}}
\end{itemize}
\end{frame}

\begin{frame}
\frametitle{Übungsbetrieb}
Es gibt Übungsblätter:
\begin{itemize}
\item Übungsblattabgabe in Zweiergruppen
\item Nur falls im gleichen Tutorium! \pause
\item Ausgabe Mittwochs, Abgabe Freitag 9 Tage später \pause
\item \textbf{Immer mit dem gleichen Partner abgeben!} \pause
\item \textbf{Plagiate werden mit 0 Punkten bewertet!}
\end{itemize}
\end{frame}

\begin{frame}
\frametitle{Beschriftung}
\begin{itemize}
\item Namen und Matrikelnummern (bei Partnerabgabe für beide)
\item Nummer des Übungsblatts, Name des Tutors
\item rechts oben groß: Nummer des Tutoriums! (mein Tut: 9)
\item gut zusammenheften
\item \textbf{Zuwiderhandelnde haben keinen Punktanspruch!}
\end{itemize}
\end{frame}

\begin{frame}
\frametitle{Vorrechnen an der Tafel}
\begin{itemize}
\item Vorrechnen von Aufgaben des letzten Übungsblatts
\item Im Mittel zwei Studenten pro Woche
\item Jeder darf maximal einmal vorrechnen (erstes Mal zählt)
\item Gibt maximal 6 Übungspunkte
\end{itemize}
\end{frame}

\begin{frame}
\frametitle{Was bringt das euch?}
\begin{itemize}
\item Punkte aus Übungsblättern
\item + Punkte aus Mittsemesterklausur (Umfang etwa 2 ÜB)
\item + Punkte aus dem Vorrechnen
\item + \textbf{Zusatz-}Punkte aus Zusatzaufgaben
\item $\Rightarrow$ Bonuspunkte auf bestandene Klausur
\item (25\%$\to$1 Punkt, 50\%$\to$2, 75\%$\to$3) \pause
\item Übungsblätter sind gute Übung für Klausur
\item $\Rightarrow$ Blätter auch unabhängig vom Bonus machen
\end{itemize}
\end{frame}

\begin{frame}
\frametitle{zum Tutorium}
\begin{itemize}
\item Hilfe bei Verständnisproblemen \pause
\item angeleitetes Aufgabenlösen \pause
\item Fragen/Vorschläge/Anmerkungen willkommen!
\end{itemize}
\end{frame}

\begin{frame}
Was ist ein Algorithmus?
\end{frame}

\begin{frame}
\frametitle{Pseudocode}
\begin{itemize}
\item Vereinfachte Programmiersprache \pause
\item if, else, while, repeat \dots until, for, \dots \pause
\item Blöcke durch Einrückung (vgl. Java: "\{" und "\}") \pause
\item Zuweisung mit ":=", Kommentar "//"
\item Tupelschreibweise: (c, s) = a+b+c \pause
\item in Assertions beliebige Mathe-Ausdrücke: $\{i\geq 2\wedge \neg\exists a,b\geq 2: i=a\cdot b\}$
\end{itemize}
\end{frame}

\begin{frame}
\frametitle{Wiederholung: Die O-Notation}
\begin{itemize}
\item $O(f(n))=\{g(n):\exists c>0: \exists n_0\in\N^+: \forall n\geq n_0: g(n)\leq c\cdot f(n)\}$
\item "$g$ wächst höchstens so schnell wie $f$" \pause
\item Oft schreibt man statt z.B. $2n\in O(n)$ auch $2n=O(n)$ (ist allerdings eher unschön) \pause
\item $o(f(n))=\{g(n):\forall c>0: \exists n_0\in\N^+: \forall n\geq n_0: g(n)< c\cdot f(n)\}$
\item "$g$ wächst langsamer als $f$" \pause
\item Außerdem: $\Omega$ (mindestens), $\omega$ (schneller), $\Theta$ (genau so schnell) \pause
\item In dieser Vorlesung untersuchen wir oft den "worst case"
\end{itemize}
\end{frame}

\begin{frame}[fragile]
\frametitle{Beispiel: Bubble-Sort}
\begin{lstlisting}
  function bubbleSort(a[0..n-1])
  for j:=0 to n-1
    for i:=0 to n-1
      if a[i] > a[i+1]
        (a[i],a[i+1]) := (a[i+1],a[i]) 
\end{lstlisting}%\pause
%Laufzeit ist $\Theta(n^2)$. Geht das noch besser?
\end{frame}

\begin{frame}[fragile]
\frametitle{Beispiel: Bubble-Sort}
\begin{lstlisting}
  function bubbleSort2(a[0..n-1])
    repeat
      swapped := false
      for i:=0 to n-1
        if a[i] > a[i+1]
          (a[i],a[i+1]) := (a[i+1],a[i]) 
          swapped := true
    until swapped = false
\end{lstlisting}%\pause
%Laufzeit ist im worst (und average) case $\Theta(n^2)$, im best case aber $\Theta(n)$.
\end{frame}

\begin{frame}
\frametitle{Aufgaben zum O-Kalkül}
Zeige oder widerlege:
\begin{enumerate}
\item $10n^2 + 5n^2 \in \omega(n^2)$
\item $\log_a(n)\in\Theta(\log_b(n))$ ($a,b>1$)
\item $n^2 \in o(n^2 \log(n))$
\item $(\log(n))^5 \in o(n)$
\item $2^n \in o(n^2)$
\item $n! \in \omega(2^n)$
\end{enumerate}
\end{frame}

\begin{frame}
\frametitle{Das Master-Theorem}
Gegeben sei eine Rekurrenz-Gleichung der Form $T(n) = a \cdot T(\frac{n}{b}) + f(n)$.\\
Es gibt nun drei Fälle:
\begin{enumerate}
\item $f(n) \in \mathcal{O}\left( n^{\log_b a - \varepsilon} \right)$ 
für ein $\varepsilon>0$\\ $\Rightarrow$ $T(n) \in \Theta\left( n^{\log_b a} \right)$ \pause
\item $f(n) \in \Theta\left( n^{\log_b a} \right)$\\ $\Rightarrow T(n)$ $\in \Theta\left( n^{\log_b a} \log(n)\right)$ \pause
\item $f(n) \in \Omega\left( n^{\log_b a + \varepsilon} \right)$ für ein $\varepsilon>0$
und ebenfalls für ein $c$ mit $0 < c < 1$ und alle hinreichend großen $n$ gilt:
$a \cdot f( \textstyle \frac{n}{b} ) \leq c f(n)$\\ $\Rightarrow$ $T(n) \in \Theta(f(n))$
\end{enumerate}
\end{frame}

\begin{frame}[fragile]
\frametitle{Algorithmus von Karatsuba-Ofman}
\begin{lstlisting}
  function rectMult(a, b $\in$ $\N$)
    assert a und b haben n = 2k Ziffern
    if n = 1 then return a $\cdot$ b
    Schreibe a als $a_1$ $\cdot$ $B^k$ + $a_0$
    Schreibe b als $b_1$ $\cdot$ $B^k$ + $b_0$
    $c_{11}$ := recMult($a_1$,$b_1$)
    $c_{00}$ := recMult($a_0$,$b_0$)
    $c_{0110}$ := recMult($a_1$ + $a_0$,$b_1$ + $b_0$)
    e := $c_{11}$ $\cdot$ $B^{2k}$ + ($c_{0110}$ - $c_{11}$ - $c_{00}$) $\cdot$ $B^k$ + $c_{00}$
    return e
\end{lstlisting}%\pause
%Mit Mastertheorem folgt: Die Laufzeit ist in $\Theta(n^{\log_2(3)})\approx\Theta(n^{1.58})$
\end{frame}

\begin{frame}[fragile]
\frametitle{Beispiel: Lineare Suche}
\begin{lstlisting}
  function linearSearch(a[0..n-1], z)
    for i:=0 to n-1
      if a[i]=z
      	return true
    return false
\end{lstlisting}%\pause
%worst/average: $\Theta(n)$, best: $\Theta(1)$
\end{frame}

\begin{frame}[fragile]
\frametitle{Beispiel: Binäre Suche}
\begin{lstlisting}
  function binarySearch(a[0..n-1], z)
    assert a ist sortiert
    (l,r):=(0,n-1)
    while l$\leq$r
      m:=$\lfloor \frac{l+r}{2}\rfloor$
      if a[m]=z then return true
      if a[m]>z then r:=m-1 else l:=m+1
    return false
\end{lstlisting}%\pause
%worst/average: $\Theta(\log(n))$, best: $\Theta(1)$
\end{frame}

\begin{frame}[fragile]
\frametitle{Beispiel: Bogosort}
Hier die Grobstruktur:
\begin{lstlisting}
  function bogosort(a[0..n-1])
    while a ist nicht sortiert
      ordne Elemente von a per Zufall an
\end{lstlisting}%\pause
%best case: $\Theta(n)$, average case (laut Wikipedia): $Theta(n\cdot n!), und der worst case ist $\infty$!
\end{frame} 
\end{document}