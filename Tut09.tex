\documentclass{beamer}

\usepackage{sdq/templates/beamerthemekit}

\usepackage[utf8]{inputenc}
\usepackage[ngerman]{babel}
\usepackage[TS1,T1]{fontenc}
\usepackage{array}
\usepackage{xspace}
\usepackage{xcolor}
\usepackage{listings}
\usepackage[scaled]{beramono}

\beamertemplatenavigationsymbolsempty

% vertical table padding
\renewcommand{\arraystretch}{1.5}

\definecolor{colkeywords}{rgb}{0,0,0.4}
\definecolor{colcomment}{rgb}{0,0.5,0}
\definecolor{colstring}{rgb}{0.5,0,0}
\definecolor{colconst}{rgb}{0.9,0.1,0.8}
\definecolor{collinenums}{rgb}{0.5,0.5,0.5}
\definecolor{coldigit}{rgb}{0.9,0.1,0.8}

\lstset{
  basicstyle=\normalsize\ttfamily,
  numbers=left,
  numberstyle=\scriptsize,
  numbersep=-5pt,
  tabsize=4,
  framexleftmargin=1mm,
  xleftmargin=10mm,
  escapeinside={\%*}{*)},
  commentstyle=\color{colcomment},
  keywordstyle=\color{colkeywords}\bfseries,
  stringstyle=\color{colstring},
  numberstyle=\color{collinenums},
  mathescape,
  literate=%
    {0}{{{\color{coldigit}0}}}1
    {1}{{{\color{coldigit}1}}}1
    {2}{{{\color{coldigit}2}}}1
    {3}{{{\color{coldigit}3}}}1
    {4}{{{\color{coldigit}4}}}1
    {5}{{{\color{coldigit}5}}}1
    {6}{{{\color{coldigit}6}}}1
    {7}{{{\color{coldigit}7}}}1
    {8}{{{\color{coldigit}8}}}1
    {9}{{{\color{coldigit}9}}}1
    {Ö}{{\"O}}1
    {Ä}{{\"A}}1
    {Ü}{{\"U}}1
    {ß}{{\ss}}2
    {ü}{{\"u}}1
    {ä}{{\"a}}1
    {ö}{{\"o}}1
}

\lstdefinelanguage{Pseudo}
{keywords={Function,return,assert,while,for,to,downto,if,then,else},%
emph={FALSE,TRUE},
emphstyle=\color{colconst},
sensitive=true,%
comment=[l]{//},%
string=[b]",%
}


\newcommand{\N}{\mathbb{N}}

%\usetheme[deutsch]{KIT}
\author{Simon Bischof (simon.bischof2@student.kit.edu)}
\title{Tutorium Algorithmen 1}
\institute{Institut für theoretische Informatik, Prof. Sanders}
\titleimage{title.png}

\begin{document}

\shorthandoff{"}
\lstset{language=Pseudo}

\begin{frame}
  \titlepage
\end{frame}

\begin{frame}
\frametitle{Übungsblatt}
\begin{itemize}
\item Bei Heapsort: aktuellen Schritt hinschreiben
\item Pointerumbiegung bei Aufgabe 2(c) explizit hinschreiben
\end{itemize}
\end{frame}

\begin{frame}
\frametitle{Programmierwettbewerb}
\begin{itemize}
\item Siehe Übungsfolien 7
\item Bei Fragen könnt ihr euch an mich wenden
\end{itemize}
\end{frame}

\begin{frame}
\frametitle{Aus der Übung: Rot-Schwarz-Bäume}
\begin{itemize}
\item Balancierte Suchbäume mit zusätzlichen Eigenschaften
\item Knoten haben Farbe: Rot oder Schwarz
\item Lokal feststellbar, welche Operationen zum Balancieren nötig sind\pause
\item Äquivalent zu (2,4)-Bäumen
\end{itemize}
\end{frame}

\begin{frame}
Graphen
\end{frame}

\begin{frame}
\frametitle{Konventionen}
\begin{itemize}
\item $G=(V,E)$, $V$ Kantenmenge, $E$ Knotenmenge\pause
\item $n=|V|$, $m=|E|$, Knoten $s,t,u,v,w,x,y,z$\pause
\item Kanten $e\in E$: Knotenpaare (gerichtet) oder 2-elementige Knotenmenge (ungerichtet)\pause
\item ungerichtet $\to$ gerichtet durch bigerichtete Graphen
\end{itemize}
\end{frame}

\begin{frame}
\frametitle{Kantenanfragen}
\begin{itemize}
\item speichere Kanten in einer Hashtabelle
\item unabhängig von restlicher Struktur
\end{itemize}
\end{frame}

\begin{frame}
\frametitle{Kantenfolgerepräsentation}
\begin{itemize}
\item Liste aller Kantenpaare\pause
\item Kompakt, gute I/O
\item Wenig sinnvolle Operationen unterstützt
\end{itemize}
\end{frame}

\begin{frame}
\frametitle{Adjazenzfeld}
\begin{itemize}
\item V=1..n oder V=0..n-1
\item Kantenfeld E speichert Ziele gruppiert nach Startknoten
\item V speichert Index der ersten ausgehenden Kante\pause
\item Dummy-Eintrag V[n+1] speichert m+1
\end{itemize}
\end{frame}

\begin{frame}
\frametitle{Adjazenzliste}
\begin{itemize}
\item Knoten-Array mit doppelt verketterter Liste von ausgehenden Kanten\pause
\item Einfaches Löschen und Einfügen von Kanten
\item platz-, cacheineffizient
\end{itemize}
\end{frame}

\begin{frame}
\frametitle{Adjazenzmatrix}
\begin{itemize}
\item A$\in\{0,1\}^{n\times n}$; A(i,j)=1 $\Leftrightarrow$ (i,j)$\in$E\pause
\item platzeffizient für sehr dichte Graphen, platzineffizient sonst
\item einfache Kantenanfragen, langsame Navigation
\item verbindet lineare Algebra und Graphentheorie
\end{itemize}
\end{frame}

\begin{frame}
\frametitle{Customization}
\begin{itemize}
\item Zuschneiden von Datenstrukturen für Algorithmen
\item Manchmal implizite Repräsentation möglich
\end{itemize}
\end{frame}

\begin{frame}
\frametitle{Kreativaufgabe}
In dieser Aufgabe soll ein dynamisiertes Adjazenzfeld entwickelt werden. Mit anderen Worten:
Gesucht ist eine Datenstruktur für gerichtete Graphen G = (V,E) mit folgenden Eigenschaften:
\begin{itemize}
\item Stabile und eindeutige KnotenIDs. Knoten sollen durch IDs eindeutig identifiziert werden.
Diese IDs sollen Zahlen aus $\mathbb{N}_0$ sein. Dabei seien die KnotenIDs stabil, d.h. die ID eines
Knotens ändere sich nie solange dieser Knoten existiert (nach Entfernen eines Knotens darf
dessen ID jedoch neu vergeben werden).
\item Eindeutige KantenIDs. Die Kanten sollen ebenfalls durch IDs eindeutig identifiziert werden.
Allerdings müssen diese nicht unbedingt Zahlen aus $\mathbb{N}_0$ sein und sie müssen auch
nicht stabil sein.
\item Effizienter Wahlfreier Zugriff auf Knoten und Kanten. Es gibt die Operationen
 node(u : NodeID) : Handle of Node und
 edge(e : EdgeID) : Handle of Edge,
die in O(1) Zeit einen Handle auf das Knoten bzw. Kantenobjekt zu einer Knoten- bzw.
Kanten-ID liefern.
\end{itemize}
\end{frame}

\begin{frame}
\begin{itemize}
\item Effiziente Navigation. Es gibt die Operationen
 firstEdge(v : NodeID) : EdgeID $\cup \{\bot\}$ und
 nextEdge(e : EdgeID) : EdgeID $\cup \{\bot\}$
mit deren Hilfe wie folgt über alle ausgehenden Kanten eines Knoten v iteriert werden
kann in einem Graph G:\\
for ( EdgeID e := graph.firstEdge(v); e $\neq\bot$; e := nextEdge(e) )\\
$h_e$ := G:edge(e) : Handle of Edge\\
/* do something */\\
end for\\
Sowohl firstEdge als auch nextEdge dürfen höchstens O(1) Zeit brauchen.
\end{itemize}
\end{frame}

\begin{frame}
\begin{itemize}
\item Amortisiert konstantes Einfügen von Knoten und Kanten. Es gibt Operationen
 insertNode : NodeID und
 insertEdge(u, v : NodeID) : EdgeID,
die in amortisiert konstanter Zeit einen neuen Knoten bzw. eine neue Kanten von u nach v
einfügen und jeweils die ID des neu erzeugten Elementes zurückliefern. Beide Operationen
dürfen höchstens amortisiert konstante Zeit kosten.
\item Amortisiert konstantes Entfernen von Knoten und Kanten. Es gibt Operationen
 deleteNode(v : NodeID) und
 deleteEdge(e : EdgeID),
die einen Knoten bzw. eine Kante entfernen. Der Einfachheit halber darf ein Knoten dabei
nur entfernt werden, wenn bereits alle seine Kanten entfernt worden sind. Beide Operationen
dürfen höchstens amortisiert konstante Zeit kosten.
\end{itemize}
\end{frame}

\begin{frame}
\begin{itemize}
\item Überlegen Sie sich, wie Sie diese Datenstruktur realisieren.
\item Begründen Sie, warum die beschriebenen Operationen in Ihrer Realisierung das geforderte
Laufzeitverhalten aufweisen.
\item Wieviel Speicher kann ein Graph mit Ihrer Realisierung im schlimmsten Fall belegen
(abhängig von aktuellen oder zwischenzeitlichen Werten von |V| und |E| und das nicht
nur im O-Kalkül)? Wieviel im besten Fall? Vergleichen Sie mit dem Speicherverbrauch
des statischen Adjazenzfeldes aus der Vorlesung.
\end{itemize}
\end{frame}

\begin{frame}
\frametitle{Graphtraversierung}
\begin{itemize}
\item Systematisches Durchsuchen eines Graphen\pause
\item Klassifizierung von Kanten als Baum-, Vorwärts-, Quer- und Rückwärtskanten
\end{itemize}
\end{frame}

\begin{frame}
\frametitle{Breitensuche}
\begin{itemize}
\item Wähle Startknoten s
\item Berechnet Abstand zu s (pro Kante Abstand 1)\pause
\item Berechnet Baum Schicht für Schicht\pause
\item Keine Vorwärtskanten (warum?)\pause
\item Speichert Distanz zu s und Vorgänger (parent[s]=s)
\end{itemize}
\end{frame}

\begin{frame}
\frametitle{Tiefensuche}
\begin{itemize}
\item Wähle Startknoten s\pause
\item versuche soweit wie möglich zu gehen\pause
\item Keine weitere Kante: Backtracken
\end{itemize}
\end{frame}

\begin{frame}
\frametitle{Markierungen etc.}
\begin{itemize}
\item Markiere besuchte Knoten\pause
\item Verwalte Zähler dfsPos, finishingTime : 1..n\pause
\item beim Besuchen: dfsNum[w]=dfsPos++
\item beim Abschließen: finishTime[w]=finishingTime++
\end{itemize}
\end{frame}

\begin{frame}
\frametitle{Klassifikation von Kanten}
\begin{tabular}{l||l|l|l}
type & dfsNum[v]< & finishTime[w]< & w is\\
(v,w) & dfsNum[w] & finishTime[v] & marked\\\hline
tree & yes & yes & no\\
forward & yes & yes & yes\\
backward & no & no & yes\\
cross & no & yes & yes\\
\end{tabular}
\end{frame}

\begin{frame}
\frametitle{DAG/Topologische Sortierung}
\begin{itemize}
\item G DAG $\Leftrightarrow$ DFS finded keine Rückwärtskante\pause
\item Dann liefert t(w):=n-finishingTime[w] eine topologische Sortierung
\end{itemize}
\end{frame}

\begin{frame}
\frametitle{starke Zusammenhangskomponenten}
\begin{itemize}
\item Starke Zusammenhangskomponenten
Betrachte die Relation $\stackrel{*}{\leftrightarrow}$ mit $u\stackrel{*}{\leftrightarrow}v$ falls $\exists$ Pfad $\langle v,\ldots,u\rangle$ und $\exists$ Pfad $\langle u,\ldots,v\rangle$
\item $\stackrel{*}{\leftrightarrow}$ ist Äquivalenzrelation (warum?)\pause
\item Die Äquivalenzklassen von $\stackrel{*}{\leftrightarrow}$ bezeichnet man als starke Zusammenhangskomponenten.
\end{itemize}
\end{frame}
\end{document}