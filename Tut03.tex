\documentclass{beamer}

\usepackage{sdq/templates/beamerthemekit}

\usepackage[utf8]{inputenc}
\usepackage[ngerman]{babel}
\usepackage[TS1,T1]{fontenc}
\usepackage{array}
\usepackage{xspace}
\usepackage{xcolor}
\usepackage{listings}
\usepackage[scaled]{beramono}

\beamertemplatenavigationsymbolsempty

% vertical table padding
\renewcommand{\arraystretch}{1.5}

\definecolor{colkeywords}{rgb}{0,0,0.4}
\definecolor{colcomment}{rgb}{0,0.5,0}
\definecolor{colstring}{rgb}{0.5,0,0}
\definecolor{colconst}{rgb}{0.9,0.1,0.8}
\definecolor{collinenums}{rgb}{0.5,0.5,0.5}
\definecolor{coldigit}{rgb}{0.9,0.1,0.8}

\lstset{
  basicstyle=\normalsize\ttfamily,
  numbers=left,
  numberstyle=\scriptsize,
  numbersep=-5pt,
  tabsize=4,
  framexleftmargin=1mm,
  xleftmargin=10mm,
  escapeinside={\%*}{*)},
  commentstyle=\color{colcomment},
  keywordstyle=\color{colkeywords}\bfseries,
  stringstyle=\color{colstring},
  numberstyle=\color{collinenums},
  mathescape,
  literate=%
    {0}{{{\color{coldigit}0}}}1
    {1}{{{\color{coldigit}1}}}1
    {2}{{{\color{coldigit}2}}}1
    {3}{{{\color{coldigit}3}}}1
    {4}{{{\color{coldigit}4}}}1
    {5}{{{\color{coldigit}5}}}1
    {6}{{{\color{coldigit}6}}}1
    {7}{{{\color{coldigit}7}}}1
    {8}{{{\color{coldigit}8}}}1
    {9}{{{\color{coldigit}9}}}1
    {Ö}{{\"O}}1
    {Ä}{{\"A}}1
    {Ü}{{\"U}}1
    {ß}{{\ss}}2
    {ü}{{\"u}}1
    {ä}{{\"a}}1
    {ö}{{\"o}}1
}

\lstdefinelanguage{Pseudo}
{keywords={Function,Procedure,return,assert,while,for,to,downto,if,then,else,Pointer,Class,invariant,this,NIL,of,Array},%
emph={FALSE,TRUE},
emphstyle=\color{colconst},
sensitive=true,%
comment=[l]{//},%
string=[b]",%
}


\newcommand{\N}{\mathbb{N}}

%\usetheme[deutsch]{KIT}
\author{Simon Bischof (simon.bischof2@student.kit.edu)}
\title{Tutorium Algorithmen 1}
\institute{Institut für theoretische Informatik, Prof. Sanders}
\titleimage{title.png}

\begin{document}

\shorthandoff{"}
\lstset{language=Pseudo}

\begin{frame}
  \titlepage
\end{frame}

\begin{frame}
\frametitle{Zum Übungsblatt}
\begin{itemize}
\item Vorsicht beim Mastertheorem
\begin{itemize}
\item Immer hinschreiben, welcher Fall
\item Immer $\varepsilon$ hinschreiben
\item Vorsicht beim Runden von $\log_a b$
\item Regularität ($\exists d\in (0,1):a f(\frac{n}{b})\leq d f(n)$) immer zeigen (und $d$ angeben)
\item Es gibt noch eine einfache Version des Mastertheorems (siehe VL)
\end{itemize}\pause
\item Beweise für Invarianten etwas ausführlicher \pause
\item Bei Aufgabe 3 kam nicht $\frac{1}{0}$, sondern die leere Summe (definiert als 0) raus
\end{itemize}
\end{frame}

\begin{frame}
\frametitle{einfach verkettete Listen}
\begin{itemize}
\item Unterschied zur doppelt verketteten Liste?
\only<2->{\item weniger Platzverbrauch, schneller}
\only<3->{\item eingeschränkte Schnittstelle, z.B. kein "remove"}
\only<4->{\item Invariante: Eingangsgrad = 1}
\end{itemize}
\end{frame}

\begin{frame}[fragile]
\frametitle{wichtige Operationen}
\begin{itemize}
\item splice
\begin{itemize}
\item Beachtet die Schnittstelle!
\item Function splice(a$^\prime$,b,t:Handle)\\ $\begin{pmatrix}a^\prime\to next\\t\to next\\b\to next\end{pmatrix}:=\begin{pmatrix}b\to next\\a^\prime\to next\\t\to next\end{pmatrix}$
\end{itemize}\pause
\item pushBack
\only<3>{\begin{itemize}
\item braucht Zeiger aufs letzte Element
\end{itemize}}
\end{itemize}
\end{frame}

\begin{frame}
\frametitle{Felder (Arrays)}
\begin{itemize}
\item $A[i]=a_i$ falls $A=\langle a_0,\ldots,a_{n-1}\rangle$
\item Beschränkte Felder (Bounded Arrays): bekannt
\item Unbeschränkte Felder (Unbounded Arrays): pushBack, popBack, size
\end{itemize}
\end{frame}

\begin{frame}
\frametitle{Unbeschränkte Felder}
\begin{itemize}
\item Hinzufügen: size++, evtl. umkopieren (falls zu voll)
\item Löschen: size-{}-, evtl. umkopieren (falls zu leer)
\item Schlechte Implementierungen brauchen für $n$ Operationen bis zu $\Theta(n^2)$ Zeit
\end{itemize}
\end{frame}

\begin{frame}[fragile]
\frametitle{Unbeschränke Felder mit teilweise ungenutztem Speicher}
\begin{lstlisting}
UArray of Element
  w:=1 : $\mathbb{N}$ //allocated
  n:=0 : $\mathbb{N}$ //actual
  invariant $n\leq w<\alpha n$ or ($n=0$ and $w\leq 2$)
  b : Array[0..w-1] of Element
  
  Operator [i:$\mathbb{N}$]:Element
    assert $0\leq i<n$
    return b[i]
  
  Function size:$\mathbb{N}$ return n
\end{lstlisting}
\end{frame}

\begin{frame}[fragile]
\frametitle{hinzufügen und löschen}
\begin{lstlisting}
Procedure pushBack(e:Element)
  if n=w
    //copy to an array of size 2n
    reallocate(2n)
  b[n]:=e
  n++
  
Procedure popBack()
  assert n>0
  n--
  if 4n$\leq$w and n>0
    reallocate(2n)
\end{lstlisting}
\end{frame}

\begin{frame}
\frametitle{Amortisierte Analyse}
\begin{itemize}
\item Durchschnittliche Dauer einer Operation in einer Folge von Operationen (im worst case!)\pause
\item Sinnvoll, falls worst case garantiert selten auftritt\pause
\item KEINE average-case-Betrachtung\pause
\item Beweis z.B. mit Kontomethode (siehe VL)\pause
\item pushBack und popBack haben eine amortisierte Laufzeit von O(1)
\end{itemize}
\end{frame}

\begin{frame}
\frametitle{Amortisierte Analyse}
\begin{itemize}
\item Unbounded Arrays\pause
\item Binärzähler: Ein Binärzähler unterstütze als einzige Operation eine Inkrementierung um 1. Der Binärzähler sei am Anfang mit 0 initialisiert. Für die Implementation der Inkrementierungen darf bloß die Operation "ändere das i-te Bit", welche konstanten Aufwand besitzt, verwendet werden. Berechnen Sie die amortisierten Kosten der Inkrementierung.\\
\only<3>{Lösung: O(1)}
\end{itemize}
\end{frame}

\begin{frame}
\frametitle{Stapel und Queues}
\begin{itemize}
\item effizient und einfach für bestimmte Operationen
\item wenig fehleranfällig\pause
\item Stack (LIFO), Queue (FIFO) und Deque (beides)
\end{itemize}
\end{frame}

\begin{frame}
\frametitle{Aufgabe zu Datenstrukturen}
Beschreiben Sie eine Datenstruktur die folgendes kann:
\begin{itemize}
\item pushBack und popBack in O(1) im Worst-Case nicht nur amortisiert.
\item Zugriff auf das k-te Element in O(log n) im Worst-Case nicht nur amortisiert.
\end{itemize}
Nehmen Sie an, dass eine Speicherallokation beliebiger Größe in O(1) geht.
\end{frame}

\begin{frame}
\frametitle{Aufgabe zu Datenstrukturen 2}
Beschreiben Sie eine Datenstruktur die folgendes kann:
\begin{itemize}
\item pushBack und popBack in O(log n) im Worst-Case nicht nur amortisiert.
\item Zugriff auf das k-te Element in O(1) im Worst-Case nicht nur amortisiert.
\end{itemize}
Nehmen Sie an, dass eine Speicherallokation beliebiger Größe in O(1) geht.
\end{frame}
\end{document}