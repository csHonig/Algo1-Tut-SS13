\documentclass{beamer}

\usepackage{sdq/templates/beamerthemekit}

\usepackage[utf8]{inputenc}
\usepackage[ngerman]{babel}
\usepackage[TS1,T1]{fontenc}
\usepackage{array}
\usepackage{xspace}
\usepackage{xcolor}
\usepackage{listings}
\usepackage[scaled]{beramono}

\beamertemplatenavigationsymbolsempty

% vertical table padding
\renewcommand{\arraystretch}{1.5}

\definecolor{colkeywords}{rgb}{0,0,0.4}
\definecolor{colcomment}{rgb}{0,0.5,0}
\definecolor{colstring}{rgb}{0.5,0,0}
\definecolor{colconst}{rgb}{0.9,0.1,0.8}
\definecolor{collinenums}{rgb}{0.5,0.5,0.5}
\definecolor{coldigit}{rgb}{0.9,0.1,0.8}

\lstset{
  basicstyle=\normalsize\ttfamily,
  numbers=left,
  numberstyle=\scriptsize,
  numbersep=-5pt,
  tabsize=4,
  framexleftmargin=1mm,
  xleftmargin=10mm,
  escapeinside={\%*}{*)},
  commentstyle=\color{colcomment},
  keywordstyle=\color{colkeywords}\bfseries,
  stringstyle=\color{colstring},
  numberstyle=\color{collinenums},
  mathescape,
  literate=%
    {0}{{{\color{coldigit}0}}}1
    {1}{{{\color{coldigit}1}}}1
    {2}{{{\color{coldigit}2}}}1
    {3}{{{\color{coldigit}3}}}1
    {4}{{{\color{coldigit}4}}}1
    {5}{{{\color{coldigit}5}}}1
    {6}{{{\color{coldigit}6}}}1
    {7}{{{\color{coldigit}7}}}1
    {8}{{{\color{coldigit}8}}}1
    {9}{{{\color{coldigit}9}}}1
    {Ö}{{\"O}}1
    {Ä}{{\"A}}1
    {Ü}{{\"U}}1
    {ß}{{\ss}}2
    {ü}{{\"u}}1
    {ä}{{\"a}}1
    {ö}{{\"o}}1
}

\lstdefinelanguage{Pseudo}
{keywords={Function,return,assert,while,for,to,downto,if,then,else},%
emph={FALSE,TRUE},
emphstyle=\color{colconst},
sensitive=true,%
comment=[l]{//},%
string=[b]",%
}


\newcommand{\N}{\mathbb{N}}

%\usetheme[deutsch]{KIT}
\author{Simon Bischof (simon.bischof2@student.kit.edu)}
\title{Tutorium Algorithmen 1}
\institute{Institut für theoretische Informatik, Prof. Sanders}
\titleimage{title.png}

\begin{document}

\shorthandoff{"}
\lstset{language=Pseudo}

\begin{frame}
  \titlepage
\end{frame}

\begin{frame}
\frametitle{Mittsemesterklausur}
\begin{itemize}
\item Nächsten Montag (statt VL)
\item Pünktlich sein!
\item Hilfsmittel: 1 DIN A4 einseitig handbeschriebener Zettel
\item Unterschrift für selbständige Anfertigung
\item Tutoriumsnummer auf die Klausur schreiben
\end{itemize}
\end{frame}

\begin{frame}
\frametitle{Zum Übungsblatt}
\begin{itemize}
\item Hashing: bei Verketten braucht Initialisierung $\Theta(|t|)$
\item Hashing: Hinweise aus letztem Tut beachten!
\item Wenn ihr eine Prozedur in Pseudocode hinschreibt: Name und Parameter angeben
\item Algorithmen kann man auch sprachlich angeben (falls nicht anders verlangt)
\item Bei Aufgabe 3 hätte Mergesort genügt
\item A3: bei Mergesort minimal 4 oder 5 Vergleiche je nach Implementierung
\item Quicksort aus VL verwenden
\end{itemize}
\end{frame}

\begin{frame}
\frametitle{Auswahl}
\begin{itemize}
\item Gegeben: Folge $s$, Zahl $k\in\mathbb{N}$
\item Gesuckt: $k$.-kleinstes Element von $s$\pause
\item Spezialfall: Median ($k=\lceil \frac{|s|}{2}\rceil$) oder allg. Quantile
\end{itemize}
\end{frame}

\begin{frame}[fragile]
\frametitle{Quickselect (Laufzeit erwartet linear)}
\begin{lstlisting}
Function select($s$ : Sequence of Element;
                $k$ : $\mathbb{N}$) : Element
  assert $|s|\geq k$
  pick $p\in s$ uniformly at random //pivot key
  $a:=\langle e\in s : e < p\rangle$
  if $|a|\geq k$ then return select($a,k$)
  $b:=\langle e\in s : e = p\rangle$
  if $|a|+|b|\geq k$ then return p
  $c:=\langle e\in s : e > p\rangle$
  return select($c,k-|a|-|b|$)
\end{lstlisting}
\end{frame}

\begin{frame}[fragile]
\frametitle{Bucketsort}
\begin{lstlisting}
Procedure KSort($s$ : Sequence of Element)
  $b := \langle \langle\rangle, \langle\rangle, \ldots, \langle\rangle \rangle$
        : Array[$0..K-1$] of Sequence of Element
  foreach $e\in s$ do
    $b$[key($e$)].pushBack($e$)
  $s:=$ concatenation of $b[0],\ldots,b[K-1]$
\end{lstlisting}
\end{frame}

\begin{frame}
\frametitle{Bucketsort}
\begin{itemize}
\item Laufzeit O($n+K$)
\item Arrayimplementierung möglich (siehe VL)
\item Ist stabil 
\end{itemize}\pause
LSD-Radixsort
\begin{itemize}
\item sortiere Zahlen mit $d$ Stellen
\item $d$-mal Bucketsort nach den Stellen 0,1,\ldots,$d-1$ (von hinten)
\item Laufzeit O($d(n+K)$)
\end{itemize}
\end{frame}

\begin{frame}
Prioritätslisten
\end{frame}

\begin{frame}
\frametitle{Prioritätslisten}
\begin{itemize}
\item Gegeben: Elemente mit Schlüsseln 
\item Verwalte Menge $M$ von Elementen mit folgenden Operationen:\pause
\item insert($e$): $M:=M\cup\{e\}$
\item deleteMin: return and remove $\min M$
\end{itemize}
\end{frame}

\begin{frame}
\frametitle{Kreativaufgabe}
Gegeben sei ein Universum von Elementen die in einer Prioritätsliste verwaltet werden sollen. Die Wichtigkeit eines Elements hägt nur von seinem Schlüssel ab, wobei es nur k verschiedene Schlüssel geben soll. Man kennt den Algorithmus, der die Prioritätsliste verwenden soll, so gut, dass man weiß, dass alle Einfügungen eine geringere Priorität als das zuletzt entnommene Element haben.\\
Finden Sie eine Datenstruktur, in die neue Elemente in O(1) eingefügt werden können, und die Entnahme armortisiert in O($\frac{k}{n}+1$) funktioniert. n sei dabei die Gesamtzahl jemals eingefügter Elemente.\\
Interessant insbesondere für n > k.
\end{frame}

\begin{frame}
\frametitle{(Min-)Heaps}
\begin{itemize}
\item Heap-Eigenschaft : Bäume (oder Wälder) mit $\forall v:parent(v)\leq v$\pause
\item Binärer Heap: Heap, Binärbaum, Höhe $\lfloor \log n\rfloor$, fehlende Blätter rechts unten\pause
\item Minimum = Wurzel
\end{itemize}
\end{frame}

\begin{frame}
\frametitle{Implizite Baum-Repräsentation}
\begin{itemize}
\item Array h[1..n]
\item Schicht für Schicht\pause
\item parent(j)=$\lfloor \frac{j}{2}\rfloor$
\item linkes Kind(j)=$2j$
\item rechtes Kind(j)=$2j+1$
\end{itemize}
\end{frame}

\begin{frame}[fragile]
\frametitle{Pseudocode}
\begin{lstlisting}
Procedure insert(e:Element)
  n++; h[n]:=e
  siftUp(n) //siehe VL
  
Function deleteMin : Element
  result = h[1] : Element
  h[1] := h[n]; n--
  siftDown(1) //siehe VL
  return result
\end{lstlisting}
\end{frame}
\end{document}