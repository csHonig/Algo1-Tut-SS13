\documentclass{beamer}

\usepackage{sdq/templates/beamerthemekit}

\usepackage[utf8]{inputenc}
\usepackage[ngerman]{babel}
\usepackage[TS1,T1]{fontenc}
\usepackage{array}
\usepackage{xspace}
\usepackage{xcolor}
\usepackage{listings}
\usepackage[scaled]{beramono}

\beamertemplatenavigationsymbolsempty

% vertical table padding
\renewcommand{\arraystretch}{1.5}

\definecolor{colkeywords}{rgb}{0,0,0.4}
\definecolor{colcomment}{rgb}{0,0.5,0}
\definecolor{colstring}{rgb}{0.5,0,0}
\definecolor{colconst}{rgb}{0.9,0.1,0.8}
\definecolor{collinenums}{rgb}{0.5,0.5,0.5}
\definecolor{coldigit}{rgb}{0.9,0.1,0.8}

\lstset{
  basicstyle=\normalsize\ttfamily,
  numbers=left,
  numberstyle=\scriptsize,
  numbersep=-5pt,
  tabsize=4,
  framexleftmargin=1mm,
  xleftmargin=10mm,
  escapeinside={\%*}{*)},
  commentstyle=\color{colcomment},
  keywordstyle=\color{colkeywords}\bfseries,
  stringstyle=\color{colstring},
  numberstyle=\color{collinenums},
  mathescape,
  literate=%
    {0}{{{\color{coldigit}0}}}1
    {1}{{{\color{coldigit}1}}}1
    {2}{{{\color{coldigit}2}}}1
    {3}{{{\color{coldigit}3}}}1
    {4}{{{\color{coldigit}4}}}1
    {5}{{{\color{coldigit}5}}}1
    {6}{{{\color{coldigit}6}}}1
    {7}{{{\color{coldigit}7}}}1
    {8}{{{\color{coldigit}8}}}1
    {9}{{{\color{coldigit}9}}}1
    {Ö}{{\"O}}1
    {Ä}{{\"A}}1
    {Ü}{{\"U}}1
    {ß}{{\ss}}2
    {ü}{{\"u}}1
    {ä}{{\"a}}1
    {ö}{{\"o}}1
}

\lstdefinelanguage{Pseudo}
{keywords={Function,Procedure,return,assert,while,for,to,downto,if,then,else,Pointer,Class,invariant,this,NIL,of,Array},%
emph={FALSE,TRUE},
emphstyle=\color{colconst},
sensitive=true,%
comment=[l]{//},%
string=[b]",%
}


\newcommand{\N}{\mathbb{N}}

%\usetheme[deutsch]{KIT}
\author{Simon Bischof (simon.bischof2@student.kit.edu)}
\title{Tutorium Algorithmen 1}
\institute{Institut für theoretische Informatik, Prof. Sanders}
\titleimage{title.png}

\begin{document}

\shorthandoff{"}
\lstset{language=Pseudo}

\begin{frame}
  \titlepage
\end{frame}

\begin{frame}
\frametitle{Übungsblatt}
\begin{itemize}
\item Aufgabenstellung lesen!
\item "Gegenkanten" von Baumkanten sind auch Rückwärtskanten
\end{itemize}
\end{frame}

\begin{frame}
\frametitle{Kreativaufgabe}
Betrachte eine Menge von Währungen C mit einem Umtauschkurs von $r_{ij}$ (man erhält $r_{ij}$ Einheiten von Währung j für eine Einheit von Währung i). Eine Währungs-Arbitrage ist möglich, wenn es eine Folge von elementaren Umtauschoperationen (Transaktionen) gibt, die mit einer Einheit einer Währung beginnt, und mit mehr als einer Einheit derselben Währung endet.\\
Beschreibe einen Algorithmus, mit dem man für eine gegebenen Umtauschmatrix bestimmen kann, ob Währungs-Arbitrage möglich ist. Beweise die Korrektheit des Algorithmus.\\
Hinweis: log(xy) = log x + log y, log(1) = 0, siehe auch Buch Seite 207
\end{frame}

\begin{frame}
\frametitle{Minimale Spannbäume (MST)}
Gegeben:
\begin{itemize}
\item ungerichteter (zusammenhängender) Graph (V,E)
\item Kanten als zweielementige Teilmengen $e\subseteq E$
\item Kantengewichte c(e)$\in\mathbb{R}_+$
\end{itemize}\pause
Finde Baum (V,T) mit minimalem Gesamtgewicht, der alle Knoten verbindet
\end{frame}

\begin{frame}
\frametitle{Auswahl der Kanten}
Schnitteigenschaft
\begin{itemize}
\item Sei $S\subseteq V$ beliebig
\item Betrachte die Schnittkanten $C=\{\{u,v\}\in E:u\in S,v\in V\setminus S\}$\pause
\item Die leichteste Kante in C kann in einem MST verwendet werden
\end{itemize}\pause
Kreiseigenschaft: Die schwerste Kante auf einem Kreis wird nicht für einen MST benötigt
\end{frame}

\begin{frame}[fragile]
\frametitle{Jarník-Prim}
\begin{lstlisting}
T := $\emptyset$
S := {s} for arbitrary start node s
repeat n-1 times
  find (u,v) fulfilling the cut property for S
  S := S $\cup$ {v}
  T := T $\cup$ {(u,v)}
\end{lstlisting}
\end{frame}

\begin{frame}
\frametitle{Laufzeit}
\begin{itemize}
\item ähnlich Dijkstra
\item O((m+n)log n) mit binären Heaps
\item O(m+n log n) mit Fibonacci Heaps\pause
\item Wichtigster Unterschied: monotone PQs reichen nicht
\end{itemize}
\end{frame}

\begin{frame}[fragile]
\frametitle{Kruskal}
\begin{lstlisting}
T := $\emptyset$
foreach (u,v)$\in$E in ascending order of weight do
  if u and v are in different subtrees of (V,T)
  T := T $\cup$ {(u,v)}
return T
\end{lstlisting}
\end{frame}

\begin{frame}
\frametitle{Union-Find-Datenstruktur}
Verwalte Partition der Menge 1..n, d.h. Mengen (Blocks) $M_1,\ldots,M_k$ mit
\begin{itemize}
\item $M_1\cup\ldots\cup M_k=1..n$
\item $\forall i\neq j:M_i\cap M_j=\emptyset$
\end{itemize}\pause
Jede Menge hat einen Repräsentanten
\end{frame}

\begin{frame}[fragile]
\frametitle{Pseudocode}
\begin{lstlisting}
Class UnionFind(n:$\mathbb{N}$)
  Procedure union(i,j:1..n)
  Function find(i:1..n):1..n
\end{lstlisting}
\end{frame}

\begin{frame}
\frametitle{Implementierungen}
\begin{itemize}
\item Bäume mit parent-Zeigern: find im worst case in $\Theta(n)$\pause
\item Pfadkompression (bei find: parent von allen durchlaufenen Knoten auf Repräsentanten umbiegen): find amortisiert in O(log n)\pause
\item Union-By-Rank (Rank = Tiefe; halte damit den Baum einigermaßen balanciert): find in O(log n)
\end{itemize}
\end{frame}

\begin{frame}
\frametitle{Pfadkompression + Union-By-Rank}
\begin{itemize}
\item m find- und n link-Operationen brauchen O($m\alpha_T(m,n)$)
\item $\alpha_T$ ist die inverse Ackermannfunktion (welche SEHR schnell wächst)
\item $\alpha_T(m,n)\in\omega(1)$ aber $\leq$ 4 für "sinnvolle" m,n
\end{itemize}
\end{frame}

\begin{frame}
\frametitle{Laufzeit Kruskal}
\begin{itemize}
\item Durch Kantensortierung O(m log m)
\item Besser bei ganzzahligen Kantengewichten
\end{itemize}
\end{frame}

\begin{frame}
\frametitle{Vergleich Jarník-Prim vs. Kruskal}
Pro Jarník-Prim
\begin{itemize}
\item Asymptotisch gut für alle m,n
\item Sehr schnell für m$\gg$n
\end{itemize}
Pro Kruskal
\begin{itemize}
\item Gut für m$\in$ O(n)
\item Braucht nur Kantenliste
\item Profitiert von schnellen Sortierern (ganzzahlig, parallel,\ldots)
\item Verfeinerungen auch gut für große $\frac{m}{n}$
\end{itemize}
\end{frame}
\end{document}