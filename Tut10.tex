\documentclass{beamer}

\usepackage{sdq/templates/beamerthemekit}

\usepackage[utf8]{inputenc}
\usepackage[ngerman]{babel}
\usepackage[TS1,T1]{fontenc}
\usepackage{array}
\usepackage{xspace}
\usepackage{xcolor}
\usepackage{listings}
\usepackage[scaled]{beramono}

\beamertemplatenavigationsymbolsempty

% vertical table padding
\renewcommand{\arraystretch}{1.5}

\definecolor{colkeywords}{rgb}{0,0,0.4}
\definecolor{colcomment}{rgb}{0,0.5,0}
\definecolor{colstring}{rgb}{0.5,0,0}
\definecolor{colconst}{rgb}{0.9,0.1,0.8}
\definecolor{collinenums}{rgb}{0.5,0.5,0.5}
\definecolor{coldigit}{rgb}{0.9,0.1,0.8}

\lstset{
  basicstyle=\normalsize\ttfamily,
  numbers=left,
  numberstyle=\scriptsize,
  numbersep=-5pt,
  tabsize=4,
  framexleftmargin=1mm,
  xleftmargin=10mm,
  escapeinside={\%*}{*)},
  commentstyle=\color{colcomment},
  keywordstyle=\color{colkeywords}\bfseries,
  stringstyle=\color{colstring},
  numberstyle=\color{collinenums},
  mathescape,
  literate=%
    {0}{{{\color{coldigit}0}}}1
    {1}{{{\color{coldigit}1}}}1
    {2}{{{\color{coldigit}2}}}1
    {3}{{{\color{coldigit}3}}}1
    {4}{{{\color{coldigit}4}}}1
    {5}{{{\color{coldigit}5}}}1
    {6}{{{\color{coldigit}6}}}1
    {7}{{{\color{coldigit}7}}}1
    {8}{{{\color{coldigit}8}}}1
    {9}{{{\color{coldigit}9}}}1
    {Ö}{{\"O}}1
    {Ä}{{\"A}}1
    {Ü}{{\"U}}1
    {ß}{{\ss}}2
    {ü}{{\"u}}1
    {ä}{{\"a}}1
    {ö}{{\"o}}1
}

\lstdefinelanguage{Pseudo}
{keywords={Function,return,assert,while,for,to,downto,if,then,else},%
emph={FALSE,TRUE},
emphstyle=\color{colconst},
sensitive=true,%
comment=[l]{//},%
string=[b]",%
}


\newcommand{\N}{\mathbb{N}}

%\usetheme[deutsch]{KIT}
\author{Simon Bischof (simon.bischof2@student.kit.edu)}
\title{Tutorium Algorithmen 1}
\institute{Institut für theoretische Informatik, Prof. Sanders}
\titleimage{title.png}

\begin{document}

\shorthandoff{"}
\lstset{language=Pseudo}

\begin{frame}
  \titlepage
\end{frame}

\begin{frame}
\frametitle{Übungsblatt}
Aufgabenstellung lesen!
\begin{itemize}
\item Rang in (a,b)-Bäumen
\item Korrektheit begründen
\end{itemize}
\end{frame}

\begin{frame}
\frametitle{Kreativaufgabe}
Gegeben sei ein gerichteter, stark zusammenhängender Graph in folgender Darstellung:
\begin{itemize}
\item Ein Knotenarray, das zu jedem Knoten v einen Eintrag mit seiner ID und einen Zeiger auf ein
Array mit den von v ausgehenden Kanten enthält. Die Knoten haben eindeutige IDs.
\item Das Kantenarray mit den ausgehenden Kanten von v enthält für jede Kante e einen Eintrag mit
ihrer ID und der ID des Zielknotens der Kante.
\end{itemize}
Auf diesem Graphen soll nun eine BFS ausgeführt werden, wobei zu jedem Zeitpunkt nur O(1) zusätzlicher
Speicher verwendet werden soll. Die Laufzeit darf dabei schlechter als bei der üblichen BFS sein.
Die Art der Darstellung des Graphen soll während der BFS erhalten bleiben, es ist jedoch erlaubt z.B.
die Knoten im Knotenarray zu permutieren.
\end{frame}

\begin{frame}
\frametitle{Kreativaufgabe (Forts.)}
Geben Sie eine Pseudocode-Implementierung der BFS an die diese Bedingungen erfüllt und begründen
Sie warum diese Implementierung nur O(1) zusätzlichen Speicher benötigt. Es soll dabei für jeden
Knoten eine unbekannte Funktion f aufgerufen werden, die als Eingabe die Knoten-ID und die Ebene
(Entfernung zum Startknoten) des Knotens hat. Es kann angenommen werden, dass f während der
Ausführung O(1) und nach der Ausführung keinen Speicher benötigt.
\end{frame}

\begin{frame}
\Huge{Kürzeste Wege}
\end{frame}

\begin{frame}
\frametitle{Problemstellung}
\begin{itemize}
\item Gegeben seien ein Graph $G=(V,E)$, eine Kosten- oder Kantengewichts-Funktion $c:E\to \mathbb{R}$ und ein Startknoten $s\in V$
\item Gesucht sei für alle $v\in V$ die Länge $\mu(v)$ des kürzesten Pfades von $s$ nach $v$
\item oft auch der Pfad selbst
\end{itemize}
\end{frame}

\begin{frame}
\frametitle{Grundlegendes}
\begin{itemize}
\item Gibt es immer einen kürzesten Pfad? \visible<2->{Nein, nicht bei negativen Kreisen.}
\visible<3->{\item Einfache Lösung, falls $c(e)\equiv c\geq 0$ für alle $e\in E$?} \visible<4->{Breitensuche}
\visible<5->{\item Einfache Lösung für DAGs?} \visible<6->{Topologische Sortierung nutzen.}
\end{itemize}
\end{frame}

\begin{frame}
\frametitle{Dijkstras Algorithmus}
Datenstrukturen, Initialisierung:
\begin{itemize}
\item d[v]:= vorläufige Distanz von s nach v (d[v]$\geq\mu$(v))
\item parent[v]:= Vorgänger von v auf vorläufigem kürzesten Pfad
\visible<2->{\item für $v\in V\setminus\{s\}$: d[v]:=$\infty$, parent[v]:=$\bot$
\item d[s]:=0; parent[s]:=s}
\end{itemize}
\end{frame}

\begin{frame}
\frametitle{Relaxierung einer Kante (u,v)}
\begin{itemize}
\item falls d[u] + c(u,v) $\leq$ d[v]: setze d[v] := d[u] + c(u,v) und parent[v]:=u\pause
\item Distanzen und parent ändern sich i.A. mehrmals\pause
\item Distanz wird dabei nie länger
\end{itemize}
\end{frame}

\begin{frame}[fragile]
\frametitle{Algorithmus}
\begin{lstlisting}
initialize d, parent
all nodes are non-scanned
while $\exists$ non-scanned node u with d[u]<$\infty$
    u:=non-scanned node v with minimal d[v]
    relax all edges (u,v) out of u
    u is scanned now
\end{lstlisting}
\end{frame}

\begin{frame}
\frametitle{Implementierung}
\begin{itemize}
\item verwende adressierbare Prioritätsliste
\item Schlüssel ist d[v]\pause
\item "Dijkstra = BFS mit PQ statt FIFO"
\end{itemize}
\end{frame}

\begin{frame}
\frametitle{Korrektheit}
\begin{itemize}
\item Nur für nichtnegative Kantengewichte garantiert!\pause
\item Zeige: jeder erreichbare Knoten wird gescannt...
\item und für jeden gescannten Knoten stimmt die Entfernung
\end{itemize}
\end{frame}

\begin{frame}
\frametitle{Laufzeit}
m-Mal decreaseKey und jeweils n-Mal insert und deleteMin\pause
\begin{itemize}
\item Dijkstra (normale Arrays): O(m+n$^2$)
\item Binärer Heap (Prioritätsliste): O((m+n)log n)
\item Fibonacci-Heaps: O(m+n log n)\pause
\item Monotone ganzzahlige Prior.: O(n+m)
\end{itemize}
\end{frame}

\begin{frame}
\frametitle{Bellman-Ford}
\begin{itemize}
\item relaxiere alle Kanten (n-1)-mal
\item funktioniert auch mit negativen Kantengewicht
\item erkennt negative Kreise\pause
\item Laufzeit O(mn)
\end{itemize}
\end{frame}

\end{document}