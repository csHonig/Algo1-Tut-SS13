\documentclass{beamer}

\usepackage{sdq/templates/beamerthemekit}

\usepackage[utf8]{inputenc}
\usepackage[ngerman]{babel}
\usepackage[TS1,T1]{fontenc}
\usepackage{array}
\usepackage{xspace}
\usepackage{xcolor}
\usepackage{listings}
\usepackage[scaled]{beramono}

\beamertemplatenavigationsymbolsempty

% vertical table padding
\renewcommand{\arraystretch}{1.5}

\definecolor{colkeywords}{rgb}{0,0,0.4}
\definecolor{colcomment}{rgb}{0,0.5,0}
\definecolor{colstring}{rgb}{0.5,0,0}
\definecolor{colconst}{rgb}{0.9,0.1,0.8}
\definecolor{collinenums}{rgb}{0.5,0.5,0.5}
\definecolor{coldigit}{rgb}{0.9,0.1,0.8}

\lstset{
  basicstyle=\normalsize\ttfamily,
  numbers=left,
  numberstyle=\scriptsize,
  numbersep=-5pt,
  tabsize=4,
  framexleftmargin=1mm,
  xleftmargin=10mm,
  escapeinside={\%*}{*)},
  commentstyle=\color{colcomment},
  keywordstyle=\color{colkeywords}\bfseries,
  stringstyle=\color{colstring},
  numberstyle=\color{collinenums},
  mathescape,
  literate=%
    {0}{{{\color{coldigit}0}}}1
    {1}{{{\color{coldigit}1}}}1
    {2}{{{\color{coldigit}2}}}1
    {3}{{{\color{coldigit}3}}}1
    {4}{{{\color{coldigit}4}}}1
    {5}{{{\color{coldigit}5}}}1
    {6}{{{\color{coldigit}6}}}1
    {7}{{{\color{coldigit}7}}}1
    {8}{{{\color{coldigit}8}}}1
    {9}{{{\color{coldigit}9}}}1
    {Ö}{{\"O}}1
    {Ä}{{\"A}}1
    {Ü}{{\"U}}1
    {ß}{{\ss}}2
    {ü}{{\"u}}1
    {ä}{{\"a}}1
    {ö}{{\"o}}1
}

\lstdefinelanguage{Pseudo}
{keywords={Function,Procedure,return,assert,while,for,to,downto,if,then,else,Pointer,Class,invariant,this,NIL,of,Array},%
emph={FALSE,TRUE},
emphstyle=\color{colconst},
sensitive=true,%
comment=[l]{//},%
string=[b]",%
}


\newcommand{\N}{\mathbb{N}}

%\usetheme[deutsch]{KIT}
\author{Simon Bischof (simon.bischof2@student.kit.edu)}
\title{Tutorium Algorithmen 1}
\institute{Institut für theoretische Informatik, Prof. Sanders}
\titleimage{title.png}

\begin{document}

\shorthandoff{"}
\lstset{language=Pseudo}

\begin{frame}
  \titlepage
\end{frame}

\begin{frame}
\frametitle{Zum Übungsblatt}
\begin{itemize}
\item Achtet bitte auf formale Korrektheit
\begin{itemize}
\item Besonders bei O-Notations-Beweisen ...
\item ... und vollständiger Induktion (aktuelles Blatt)
\end{itemize} \pause
\item Schaut euch nochmal den Pseudocode an\pause
\item Beachtet die Aufgabenstellung!
\end{itemize}
\end{frame}

\begin{frame}
\frametitle{Wiederholung: Vollständige Induktion}
\begin{itemize}
\item Induktionsanfang (meist $n=0$ oder $n=1$)
\item Induktionsschritt ($n\rightsquigarrow n+1$)
\end{itemize}
\end{frame}

\begin{frame}
\frametitle{Wiederholung: Schleifeninvarianten}
Beweis ähnlich zu vollständiger Induktion
\begin{itemize}
\item Invariante gilt am Anfang
\item Invariante bleibt bei einem Durchlauf (auch dem letzten!) erhalten
\item Aus Invariante und Abbruchbedingung folgt die Nachbedingung
\end{itemize}\pause
Schleifeninvarianten beweisen nur die Korrektheit bei Terminierung!
\end{frame}

\begin{frame}
\frametitle{Rekurrenzen}
$T(n)=\begin{cases}1 & (n\leq n_0)\\ 3T(\lfloor\frac{n}{5}\rfloor+7)+cn^2 & (n>n_0)\end{cases}$\\[2em]
\only<2>{Lösung: $\Theta(n^2)$}
\end{frame}

\begin{frame}
\frametitle{Rekurrenzen}
\begin{itemize}
\item $T(n) = T(\lceil \frac{n}{4}\rceil)+T(\lfloor\frac{3n}{4}\rfloor)+n$, $T(1)=1$
\item Lösung durch Auffalten des Rekursionsbaums\pause
\item Und mit $T(n) = \begin{cases}1 & (n\leq 32) \\
T(\lceil \frac{n}{4}\rceil)+T(\lfloor\frac{3n}{4}\rfloor+5)+n & (n>32)\end{cases}${\Huge ?}
\only<3>{\item Lösung für beide Rekurrenzen ist: $\Theta(n \log n)$}
\end{itemize}
\end{frame}

\begin{frame}
\frametitle{Rekurrenz (eine letzte)}
\begin{itemize}
\item $T(n) = 2T(\lfloor\sqrt{n}\rfloor)+\log n$, $T(1)=1$\pause
\item Substituieren hilft!
\only<3>{\item $T(n)\in O(\log n \cdot \log\log n)$}
\end{itemize}
\end{frame}

\begin{frame}
\frametitle{Einschub: Graphen}
\begin{itemize}
\item Wichtige Begriffe: Knoten, Kanten, Relationen, (un)gerichtete Graphen, Knotengrade, Kantengewichte, knoteninduzierte Teilgraphen, Schlingen, Pfade, Kreise\pause
\item Wichtige Graphen sind z.B. $K_3, K_4, K_5, K_{3,3}$.\pause
\item Spezialfall Bäume: Zusammenhang, Bäume, Wurzeln, Wälder, Kinder, Eltern, ...
\end{itemize}
\end{frame}

\begin{frame}
\frametitle{DAGs}
\begin{itemize}
\item "Directed acyclic graph": Gerichteter kreisfreier Graph
\item Test: Iterativ Knoten mit Ausgangsgrad 0 entfernen; \\ G ist DAG $\Leftrightarrow$ Graph am Ende leer
\item Test läuft mit passender Datenstruktur in $O(|V|+|E|)$
\item Liefert auch topologische Sortierung
\end{itemize}
\end{frame}

\begin{frame}
Datenstrukturen (für Folgen)
\end{frame}

\begin{frame}[fragile]
\frametitle{Doppelt verkettete Listen}
\begin{lstlisting}
Class Handle = Pointer to Item

//one cell in a doubly linked list
Class Item of Element
e : Element
next : Handle
prev : Handle
invariant next$\to$prev = prev$\to$next = this
\end{lstlisting}
\end{frame}

\begin{frame}
\frametitle{Dummy-Header / Wächter-Element}
\begin{itemize}
\item Führe Element mit e=$\bot$ ein, benutze es als Listenkopf
\item Zusätzlich: mache die Liste zyklisch\pause
\item Invariante erfüllt, Vermeidung von Sonderfällen;\\ aber: mehr Speicherplatz
\end{itemize}
\end{frame}

\begin{frame}[fragile]
\frametitle{Die Listenklasse}
\begin{lstlisting}
Class List of Element
// Item h is the predecessor of the first element
// and the successor of the last element.
// Pos. before any proper element
Function head : Handle; return address of h
// init to empty sequence
h = $(\bot, head, head)$ : Item

// Simple access functions
Function isEmpty : $\{0,1\}$; return h.next = head
Function first : Handle; 
     assert $\neg$isEmpty; return h.next
Function last : Handle;
     assert $\neg$isEmpty; return h.prev
\end{lstlisting}
\end{frame}

\begin{frame}[fragile]
\frametitle{splice}
\begin{lstlisting}
//Cut out $\langle a ,\ldots,b\rangle$ and insert after t
Procedure splice(a,b,t:Handle)
  assert b is not before a $\wedge$ t $\notin$ $\langle a ,\ldots,b\rangle$
  //Cut out $\langle a ,\ldots,b\rangle$
  $a^\prime :=a\to prev$
  $b^\prime :=b\to next$
  $a^\prime\to next := b^\prime$
  $b^\prime\to prev := a^\prime$

  //insert $\langle a ,\ldots,b\rangle$ after t
  $t^\prime := t\to next$
  $b\to next := t^\prime$
  $a\to prev := t$
  $t\to next := a$
  $t^\prime\to prev := b$
\end{lstlisting}
\end{frame}

\begin{frame}
\frametitle{Und nun?}
Wir benutzen eine einmal vorhandene Variable (static in Java):
\begin{itemize}
\item "freeList" enthält nicht benötigte Elemente
\item "checkFreeList" sichert, dass diese nicht leer ist
\end{itemize}\pause
Was bringt uns das?
\only<3>{Kann für splice benutzt werden!}
\end{frame}

\begin{frame}[fragile]
\frametitle{Wichtige Listenfunktionen}
\begin{lstlisting}
Procedure moveAfter(b,a : Handle) splice(b,b,a)
Procedure moveToFront(b : Handle) moveAfter(b, head)
Procedure moveToBack(b : Handle) moveAfter(b, last)

Procedure remove(b : Handle)
    moveAfter(b, freeList.head)
Procedure popFront remove(first)
Procedure popBack remove(last)

//$(\langle a,\ldots,b\rangle,\langle c,\ldots,d\rangle)\rightsquigarrow (\langle a,\ldots,b,c,\ldots,d\rangle,\langle \rangle)$
Procedure concat(L : List)
    splice(L.first, L.last, last)
//$\langle a,\ldots,b\rangle\rightsquigarrow\langle \rangle$
Procedure makeEmpty freeList.concat(this)
\end{lstlisting}
\end{frame}

\begin{frame}[fragile]
\frametitle{Nun noch das Einfügen}
\begin{lstlisting}
Function insertAfter(x:Element, a:Handle) : Handle
  //make sure freeList is nonempty.
  checkFreeList
  $a^\prime :=$ freeList.first
  moveAfter($a^\prime,a$)
  $a^\prime \to e := x$
  return $a^\prime$
  
Function insertBefore(x:Element,b:Handle) : Handle
    return insertAfter($x,b\to prev$)
Procedure pushFront(x:Element) insertAfter(x, head)
Procedure pushBack(x:Element) insertAfter(x, last)
\end{lstlisting}
\end{frame}

\begin{frame}
\frametitle{Kreativaufgabe}
Gegeben sei ein Array A[1..n] mit n Zahlen und eine Zahl x. Nun soll ein Paar (A[i],A[j]) gefunden werden mit A[i]+A[j]=x.
\begin{itemize}
\item Geben Sie eine Lösung für x=33; A=(7,15,21,14,18,3,9) an. \\ \only<2->{Lösung: A[2]=15, A[5]=18}\pause
\item Geben Sie einen effizienten Algorithmus an, der ein solches Paar in $O(n\log n)$ findet (falls es nicht existiert, soll NIL zurückgegeben werden).\pause
\item Wie sieht es aus, wenn man alle Paare finden will?
\end{itemize}
\end{frame}
\end{document}